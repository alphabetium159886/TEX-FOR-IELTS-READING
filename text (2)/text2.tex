\documentclass[10pt, a4paper, oneside]{article}
\usepackage{amsmath, amsthm, amssymb, graphicx}
\usepackage[bookmarks=true, colorlinks, citecolor=blue, linkcolor=black]{hyperref}
\usepackage[margin = 25mm]{geometry}
\usepackage{setspace}
\usepackage{listings}

\title{Title}
\date{\today}
\author{Alphabetium}
\begin{document}
\begin{spacing}{2.0}
\maketitle


\section{READING PASSAGE 1}
\subsection{PASSAGE}
In the late 19th century two books on science and religion were published within a decade of each other.
In “The Creed of Science” William Graham tried to reconcile new scientific ideas with faith. In 1881
Charles Darwin, by then an agnostic, told him: “You have expressed my inward conviction, though far more
vividly and clearly than I could have done, that the Universe is not the result of chance.”

The other book made a much bigger splash. “History of the Conflict Between Religion and Science” by John
William Draper was one of the first post-Darwinian tomes to advance the view that—as its title suggests—
science and religion are strongly antithetical. Promoted hard by its publisher, the book went through 50
printings in America and 24 in Britain and was translated into at least ten languages. Draper's bestseller told a story of antagonism that,
ever since, has been the mainstream way to see this relationship.

In “Magisteria”, his illuminating new book, Nicholas Spencer claims that this framing, more recently 
espoused by Richard Dawkins and others, is misleading. For centuries, he says, science and religion have
been “endlessly and fascinatingly entangled”. Even (or especially) those readers inclined to disagree with
him will find his narrative refreshing.

Mr Spencer works at Theos, a religious think-tank in London, and is one of Britain's most astute observers of religious affairs. 
Some conflict between science and religion is understandable, he argues, but not
inevitable. He offers an engaging tour of the intersection of religious and scientific history: from ancient science in which 
“the divine was everywhere”, to the Abbasid caliphate in Baghdad in the ninth century and Maimonides, an illustrious Jewish thinker 
of the 12th—and onwards, eventually, to artificial intelligence. Now and again he launches salvoes against ideologues on both sides.

“Medieval science” is not an oxymoron, he writes. Nor is religious rationalism. In the 11th century Berengar of Tours held that 
“it is by his reason that man resembles God.” As religious dissent spread following the Reformation, Mr Spencer says, theology helped 
incubate modern science through the propagation of doubt about institutions and the cracking open of orthodoxies. For their part, 
an emergent tribe of naturalists strove, chisel and hammer in hand, to show that creation pointed towards a creator. Exploration of nature 
was itself a form of worship.

Mr Spencer insightfully revisits the dust-ups involving Galileo, Darwin and John Scopes (prosecuted in
Tennessee in 1925 for teaching evolution). He traces the interaction of the two disciplines in often
fascinating detail. Many pioneering scientists lived in times of religious and political strife and found in “natural philosophy”, 
as pre-modern science was known, a “ministry of reconciliation”. Thomas Sprat, dean of Westminster and biographer of the Royal Society, 
opined in 1667 that, in their experiments, men “may agree, or dissent, without faction, or fierceness”. That was not always true, 
as Isaac Newton's spats with his peers showed. Still, says Mr Spencer, by supplying an arena for calmer debate that was beyond clerical 
control, “Science saved religion from itself.”
\subsection{Question}
1.What was the main argument of John William Draper's book "History of the Conflict Between Religion and Science"?\\
A. Religion and science have been entangled throughout history.\\
B. Science and religion are strongly antithetical.\\
C. Scientific ideas can be reconciled with faith.\\
D. The Universe is not the result of chance.\\

2.What is the author's opinion on the relationship between science and religion in "Magisteria"?\\
A. Science and religion are always in conflict.\\
B. Science and religion have been endlessly entangled.\\
C. Science and religion have no intersection.\\
D. Science and religion are inherently incompatible.\\

3.What does the author say about the interaction between science and religion during times of religious and political strife?\\
A. Scientists and religious leaders always disagreed.\\
B. Science saved religion from itself.\\
C. Theology helped incubate modern science.\\
D. Naturalists proved that creation pointed towards a creator.\\

4.According to the passage, what is the author's view of medieval science?\\
A. It was an oxymoron.\\
B. It was a form of worship.\\
C. It was incompatible with religion.\\
D. It had no impact on modern science.\\

5.What did William Graham and Charles Darwin agree on?\\
A. Science and religion are strongly antithetical.\\
B. The Universe is the result of chance.\\
C. New scientific ideas can be reconciled with faith.\\
D. Religion is incompatible with agnosticism.\\

Do the following statements agree with the information given in Reading Passage 1?
write\\
YES if the statement agrees with the information\\
NO if the statement contradicts the information\\
NOT GIVEN if there is no information on this\\

1.William Graham and Charles Darwin had opposite views on the relationship between science and religion.\\
2.“History of the Conflict Between Religion and Science” was a popular book when it was first published.\\
3.According to Nicholas Spencer, science and religion have always been in conflict.\\
4.The Reformation contributed to the development of modern science by promoting doubt about institutions.\\

1.Nicholas Spencer argues that science and religion have been "endlessly and fascinatingly ANSWER" throughout history.\\
2.In the ancient world, "the divine was ANSWER".\\
3.According to Berengar of Tours, "it is by his ANSWER that man resembles God."\\
4.Thomas Sprat believed that in their experiments, men "may agree, or dissent, without faction, or ANSWER".\\

\section{READING PASSAGE 2}
\subsection{PASSAGE}
The past 15 years have been tumultuous for many European countries. In the case of Spain, the period
has been marked by recessions and debt crises between 2008 and 2014, a shakedown in the party system, 
the tarnishing of numerous institutions (including the monarchy), the eruption of populist movements on the far left and right, 
the revolt of the Catalan nationalists, the global pandemic and an upsurge in the politics of intransigence and polarisation. 
The raison d'être of Michael Reid's new book, “Spain: The Trials and Triumphs of a Modern European Country”, 
is not just to identify the problems that transfix Spain today, but also to suggest a constructive way forward.

In so doing, he tackles a host of issues, ranging from immigration, the environment and the abandonment
of rural areas to women's rights and the family, the decline of bullfighting, the crisis of Catholicism, 
the media and corruption. The upshot is a lively, highly informative and nuanced portrait of contemporary
Spain. It fills a huge void in English-language books on the country; future writers will be much indebted to it.

The greatest political tremor to hit Spain since the global financial crisis of 2007-09 has been the bid for 
independence of Catalan nationalists. During the transition from dictatorship to democracy in the 1970s, 
and in the decades thereafter, Catalans were overwhelmingly supportive of regional devolution. 
All that changed with the mauling of the Catalan statute which gave the territory greater autonomy and defined 
it changed with the mauling of the Catalan statute—which gave the territory greater autonomy and defined it as a 
“nation”—by the constitutional court in 2010. Together with the economic downturn, the court's
actions led to a vertiginous rise in support for Catalan independence.

As the author points out, foreigners have tended to romanticise the Catalan struggle, but the separatists
have proved “rigid and intolerant”, even “racist”. The Catalan conundrum is inextricably linked to one of the principal 
unresolved issues of the transition: the regional autonomous communities. Though they have
achieved a great deal—overseeing education and social services, for example—they have nonetheless promoted parochialism 
and division at the expense of national commonalities and unity. For Mr Reid, the panacea is federalism, but the question 
remains as to how political opposition to such a solution, particularly on the right, would be overcome. 

His discussion of other key issues, such as the economy, “historical memory”, the judiciary, the rejuvenation of the 
Basque Country and education, is distinguished by the same even-handed yet empathetic approach. Mr Reid is especially 
good on the faults and foibles of the political system, above all the staggering lack of accountability and the notorious 
“closed lists”, which allow party bosses to decide on the candidates for a general election, thereby stifling internal debate and dissent. 

Central to Mr Reid's portrayal of contemporary Spain is the contrast between the “golden age” between 1975 and 2000 
and the period from 2000 to 2018, by which time the “shadows had all but obscured the sun”. In your reviewer's opinion, 
this is greatly overdrawn. Mr Reid tends to idealise the transition, as indicated by the claim that it was characterised 
by “little violence” when there were 665 politically motivated deaths between 1975 and 1982. In reality these years were 
marked by would-be coups, a recession, abrupt political lurches and terrorist atrocities. Similarly, the major achievements 
of the socialist governments of 1982-96 should not mask their endemic corruption, the “dirty war” against eta(the armed Basque 
separatist group) and persistent high unemployment.

Mr Reid contends that the coup d état perpetrated 100 years ago by Miguel Primo de Rivera, a general, ought to serve as a warning to 
Spain's political class”. This is far-fetched stuff, not just because state and society were vastly different back then, but also 
because the army had been usurping civilian power for years. Urged on by the authoritarian Alfonso XIII, the army influenced policy 
and pushed through reforms favourable to its interests. No such threat exists today.
\subsection{Question}
Spain has faced a series of ANSWER over the past 15 years, from economic crises to political instability, 
the rise of populist movements, and the ongoing Catalan independence bid. Michael Reid's book, 
"Spain: The Trials and Triumphs of a Modern European Country", offers a ANSWER portrait of Spain and suggests ways to move forward. 
One of the biggest challenges Spain faces is the issue of regional autonomy, which has promoted parochialism and division at the 
expense of national unity. Reid suggests that federalism may be a solution, but this faces opposition from the ANSWER. 
The political system in Spain also suffers from a lack of accountability and stifles internal debate and dissent through 
the use of closed lists. Despite these challenges, Spain has achieved a great deal, such as overseeing education and social 
services through regional autonomous communities, and improving women's rights and the environment. The book provides 
a nuanced understanding of Spain's ANSWER and successes, and is a valuable addition to English-language literature on the country. 
However, the author's portrayal of the "golden age" of Spain's transition to democracy may be overstated, as this period was also 
marked by ANSWER and political instability.

Do the following statements agree with the information given in Reading Passage 1?
write\\
TRUE if the statement agrees with the information\\
FALSE if the statement contradicts the information\\
NOT GIVEN if there is no information on this\\

The recession from 2008 to 2014 was the greatest political tremor to hit Spain since the global financial crisis of 2007-09.\\
Foreigners tend to romanticize the Catalan separatists.\\
According to the author, the socialist governments of 1982-96 had no major achievements.\\

What is the author's proposed solution to the Catalan conundrum?\\
What led to a rise in support for Catalan independence?\\
What is the author's portrayal of contemporary Spain?\\
A. The greatest political tremor to hit Spain since the global financial crisis of 2007-09 has been the bid for independence of Catalan nationalists.\\
B. For Mr Reid, the panacea is federalism, but the question remains as to how political opposition to such a solution, particularly on the right, would be overcome.\\
C. In your reviewer's opinion, this is greatly overdrawn. Mr Reid tends to idealise the transition, 
as indicated by the claim that it was characterised by “little violence” when there were 665 politically 
motivated deaths between 1975 and 1982.\\

\section{READING PASSAGE 3}
\subsection{PASSAGE}
The mayor of your city has announced a strange new public project: a lavish park especially for cats. It
seems like a waste of money so, with the help of some activists you have met online, you campaign
against it on social media. You start with rousing posts—“Breaking News: Outrageous! City prioritises elitist pets over our kids!”
—and funny memes. You soon move on to doctoring images to make it look like the mayor is part of “an ultra-secret cat-worshipping cult”. 
You galvanise your followers to take violent action.

In “Cat Park” players learn to become disinformation warriors. The free 15-minute online game explores the
dark art of spreading lies online; players get points for the passion of their posts and shareability of their memes. 
It is good fun, with a witty script and futuristic cyberpunk style. It is also an educational tool, funded by the Global 
Engagement Centre (gec), a branch of the us State Department which aims to “recognise, understand, expose and counter foreign 
state and non-state propaganda and disinformation efforts”.

Games such as “Cat Park” are an ingenious response to a widespread problem. Fake news and conspiracy
theories are in rich supply; demand for them is high in polarised countries across the world. Many governments are 
mulling policies to try to limit their spread, since internet users often struggle to discern legitimate sources from nefarious ones. 
Last year a study by Ofcom, a British regulator, found that 30$\%$ of the country's adults hardly consider the truthfulness of 
information they read online. About $6\%$ give no thought to the veracity of stories. Around a quarter 
failed to spot fake social-media accounts

Tilt Studio, the Dutch developer behind “Cat Park”, has also worked with the British government, the
European Commission and nato to create games that “help tackle online manipulation head-on”. In 2020 it
collaborated with the gec on “Harmony Square”, in which players seek to destabilise an idyllic 
neighbourhood by using falsehoods to foment disunity. During the pandemic, it released “Go Viral!”, 
a fiveminute game that gets players to scrutinise misleading information about covid-19.

“Rather than simply waiting for lies to spread, and then debunking them with a fact-check, we can leverage
games like ‘Cat Park' to practically educate ourselves about common disinformation techniques,” says Davor
Devcic of gec. Aimed primarily at citizens in the West, the games are based on the idea of “active inoculation”: 
just as individuals build up resistance to a disease after a vaccine, after playing “Cat Park” or “Harmony Square” 
they are more wary of internet skulduggery. A study by the University of Cambridge found that players of “Harmony Square” 
were better at spotting dodgy content and less likely to share it. The effect was consistent across right-wing and left-wing players.

The Canadian government, meanwhile, helped fund “Lizards and Lies”, a board game about information
warfare. It takes the form of a traditional map-based war-game, which you play as one of four characters: an “edgelord”, 
“conspiracy theorist”, “platform moderator” or “digital literacy educator”. (You are either a “spreader” or a “stopper” of lies.) 
Cards and tokens help you win over enclaves of supporters. Points are scored for each social-media network you control. It pays to 
focus on areas of the map that are winnable: as with their real-life counterparts certain online networks are more amenable to wild 
conspiracism than others.


\subsection{Question}
Match each statement with the correct description from the list below:

List of Descriptions:
A. The purpose and benefits of games like "Cat Park" and "Harmony Square"
B. The problem of fake news and conspiracy theories on the internet
C. The role of Tilt Studio in creating games to tackle online manipulation
D. The Canadian government's involvement in funding a board game about information warfare

Statements:

1.Games like "Harmony Square" have been found to help players become more aware of disinformation techniques.
2.Tilt Studio has collaborated with various governments and organizations to create games that combat online manipulation.
3.Fake news and conspiracy theories are a significant issue on the internet, with many people struggling to discern legitimate sources.
4.The Canadian government has helped fund a board game called "Lizards and Lies" about information warfare.

Do the following statements agree with the claims of the writer in Reading Passage? Write YES, NO or NOT GIVEN in boxes on your answer sheet.

1."Cat Park" is a free 15-minute online game that teaches players how to spread lies online.
2.Many governments are working to limit the spread of fake news and conspiracy theories because internet users are generally good at distinguishing between legitimate and nefarious sources.
3.According to a study by Ofcom, a British regulator, about a quarter of the country's adults failed to spot fake social media accounts.
4.The games created by Tilt Studio, a Dutch developer, are primarily intended for Western audiences, and are based on the concept of "active inoculation".
5.Players of "Harmony Square" were better at identifying misleading content and less likely to share it, according to a University of Cambridge study.

\section{READING PASSAGE 4}
\subsection{PASSAGE}
Just days after Ursula von der Leyen returned from a visit to China on April 7th, the president of the
European Commission had been due to fly to South America to nudge a trade deal along. In a small act of
mercy, the meeting with the Brazilian president has had to be postponed. She probably needs the breather.
In the last four months alone her diary has included a visit to President Joe Biden in Washington, a wellreceived 
address to the Canadian parliament, tea with King Charles near London, a guest appearance at a German cabinet meeting, 
repeat summits of the eu's 27 national heads in Brussels and trips to see the
leaders of France, Italy, Sweden, Estonia, Britain, Norway and Ukraine. Next month she will jet off to attend the g7 summit in Japan.

These jaunts are no grandstanding indulgence. The eu is in the midst of upheaval. War on the European
continent has forced a recasting of its six-decade peace project. Mrs von der Leyen is shaping the response to the challenges 
buffeting the eu, from missing Russian gas to anaemic defence spending. Its economy, just out of covid-19, is on a new set of tracks, 
the better to counter America's protectionist green subsidies, lessen Europe's over-reliance on China and deal with the imperatives of 
climate change.

Previous crises that engulfed the bloc, such as the euro-zone miasma a decade ago, had threatened to tear
the eu apart. But facing pandemic, then war, a sense of common purpose has helped to give the club a
weightiness it has rarely enjoyed. Europe has rallied around its blue-and-gold-star eu flag, a row of which flutter outside the commission's 
headquarters in Brussels. Sitting in her cavernous office on its 13th floor, Mrs von der Leyen tells The Economist: 
“We've shown this unity because we've understood from the very Apr 9th 2023 | BRUSSELS Share beginning that this Russian war in 
Ukraine will change Europe.”

Discreetly, under her leadership, the political fabric of the continent has been rewoven, with far more power flowing to the commission 
she heads. That the 64-year-old German would do so much to steer that change was once far from obvious: power in Europe is more often 
wielded by national leaders, starting with those of France and Germany. Her appointment in 2019 had come as something of a surprise. 
A longtime ally of Angela Merkel, the former chancellor from the same centre-right cdu party, she had survived rather than
thrived in the tricky defence brief for five years. The top job in Brussels was a convenient exit ramp; for the first time in five 
decades a German would sit atop the commission. It was a position the multilingual Mrs von der Leyen seemed suited to: she grew up in 
Brussels, her father having been a senior eu official in the bloc's formative years. “I'm very much born European,” she says.

Three quirks amused Eurocrats in her early days. The first was Mrs von der Leyen's unusual path to power—
she studied economics before becoming a medical doctor, then juggling a political career and seven
children. The second was her decision to turn part of her Brussels office into a studio to live in, to cram in long days and nights 
of work (some predecessors had been less diligent). The third was a habit of describing herself from the outset as heading a 
“geopolitical” commission. Running the eu's 32,000-strong executive in Brussels is more often the stuff of grinding technocracy, 
not high politics—think chemicals regulation and tweaking wheat subsidies rather than war and peace.

The claim seems less grandiose nowadays. Covid-19, which hit soon after she took office, provided an early
test. Mrs von der Leyen fought to keep barriers between eu countries from re-emerging. Her staff was tasked by national 
governments with procuring vaccines for 447m Europeans—a task it was ill-prepared for, and pulled off only after costly initial delays.

Mrs von der Leyen speaks of the commission having “to grab the opportunity and to show leadership”. One
example was a €750bn $(\$820bn) $pandemic recovery fund, a federalising leap (albeit, she stresses, a one-off
event though others might not agree about that) Cleverly the money can only be disbursed according to
event, though others might not agree about that. Cleverly, the money can only be disbursed according to
priorities set in Brussels—which has used the fund to bludgeon countries felt to fall short of eu rules.
Poland and Hungary, who are deemed to have hobbled their judiciaries, have still not seen any cash.

War on the continent catalysed further changes. The eu responded to the full-scale invasion of Ukraine by
orchestrating ten rounds of sanctions against Vladimir Putin's regime, and has delivered some €38bn in
financial assistance. It has even, in a radical departure, paid for some €3.6bn-worth of arms, once very much a taboo. 
Domestically, an energy crisis that once looked to send the eu economy into recession has
somewhat abated. 

The fighting in Ukraine—and the cutting off of Russian gas that ensued—raised questions about Europe's reliance on the outside world. 
Mrs von der Leyen speaks of “resilience”—a concept not too far removed from the “strategic autonomy” preferred by France's Emmanuel Macron, 
with whom she has just travelled to China. That trip provided an illustration of tricky power dynamics in Europe. Just before the visit Mrs 
von der Leyen had warned in a hawkish speech that “China has now turned the page on the era of ‘reform and opening' and is moving into a 
new era of security and control”. Officials in Beijing made sure she played second fiddle to the far more doveish French president, 
for example ensuring that she got less face time with President Xi Jinping. Those looking for divisions in Europe's approach to China 
found it easy to do so. 

Whether keeping its lights on, developing weapons or building electric cars, Europe increasingly wants to stand on its own two feet. 
eu rules that had kept its economies among the most open in the world, dependent on supply chains far outside its borders, are now out 
of favour. A new economic model with a far bigger role for the state—including the Brussels bureaucracy—is slowly emerging. 
In part that is a result of the only part of her original agenda to have survived contact with events: Europe is on track to reduce 
carbon emissions by $55\%$ from 1990 levels by the end of this decade, and has a plausible chance of reaching net zero by 2050.



\subsection{Question}

1.Fill in the blank: Mrs von der Leyen's diary includes visits to the leaders of France, Italy, Sweden, Estonia, Britain, Norway, and ANSWER.
\\
2.True/False/Not Given: The European Union's peace project has been reshaped due to the war on the European continent.\\

3.What is the main focus of Ursula von der Leyen's current work as president of the European Commission?\\
A. Dealing with the economic challenges facing Europe.\\
B. Preventing war on the European continent.\\
C. Encouraging European countries to rely less on China.\\
D. Increasing defense spending in Europe.\\

4.Match the following paragraphs with their main ideas:\\
(A) Paragraph 1\\
(B) Paragraph 2\\
(C) Paragraph 3\\

i. Mrs von der Leyen's appointment as president of the European Commission was unexpected.\\
ii. The EU is facing a number of challenges, but has rallied around its flag during the pandemic and war.\\
iii. Mrs von der Leyen has redefined the role of the commission and has become a powerful figure in European politics.\\

5.True/False/Not Given: \\
Mrs von der Leyen's appointment as president of the European Commission was welcomed by all EU leaders.
Ursula von der Leyen's recent travels have been unnecessary grandstanding\\

Europe has faced previous crises that have threatened to tear the EU apart.\\

Mrs. von der Leyen's appointment as head of the European Commission was expected.\\
\section{Answer}
\subsection{READING PASSAGE 1}
B;B;B;B;C;NOT GIVEB;YES;NO;YES;entangled;everywhere;reason;fierceness
\subsection{READING PASSAGE 2}
(Challenges/Trials/Successes);(Nuanced/Positive/Negative);(Left/Right/Centre);(Problems/Opportunities/Solutions);(Violence/Achievements/Corruption);
FALSE;TRUE;FALSE;B;A;C
\section{READING PASSAGE 3}
A;C;B;D;NO, NO, YES, NOT GIVEN, YES.
"Cat Park" is a free 15-minute online game that teaches players how to spread lies online.
NO. The passage mentions that "Cat Park" is a game that teaches players how to spot fake news and propaganda, not how to spread lies online.
\\
Many governments are working to limit the spread of fake news and conspiracy theories because internet users are generally good at distinguishing between legitimate and nefarious sources.
NO. The passage does not make any claim about why governments are working to limit the spread of fake news and conspiracy theories.
\\
According to a study by Ofcom, a British regulator, about a quarter of the country's adults failed to spot fake social media accounts.
YES. The passage states that "About a quarter of UK adults failed to spot fake social media accounts, according to a 2018 study by Ofcom."
\\
The games created by Tilt Studio, a Dutch developer, are primarily intended for Western audiences, and are based on the concept of "active inoculation".
NOT GIVEN. The passage mentions that Tilt Studio develops games that teach people to recognize propaganda and fake news, but it doesn't explicitly state whether their games are intended primarily for Western audiences.
\\
Players of "Harmony Square" were better at identifying misleading content and less likely to share it, according to a University of Cambridge study.
YES. The passage states that "players of 'Harmony Square' were better at identifying misleading content and less likely to share it, according to a University of Cambridge study."

\subsection{READING PASSAGE 4}

Not Given;Not Given;A;123;
NO;YES;NO
\end{spacing}{}

\end{document}