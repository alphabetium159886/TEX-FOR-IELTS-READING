\documentclass[10pt, a4paper, oneside]{article}
\usepackage{amsmath, amsthm, amssymb, graphicx}
\usepackage[bookmarks=true, colorlinks, citecolor=blue, linkcolor=black]{hyperref}
\usepackage[margin = 25mm]{geometry}
\usepackage{setspace}
\usepackage{listings}

\title{Title}
\date{\today}
\author{Alphabetium}
\begin{document}
\begin{spacing}{2.0}
\maketitle


\section{READING PASSAGE 1}

Outdoor learning is an educational approach that focuses on teaching children in outdoor environments. It has been found to 
have numerous benefits for children, including improved physical health, better academic performance, and increased engagement in learning.

One of the key benefits of outdoor learning is that it promotes physical activity. In traditional classroom settings, 
children often spend long periods of time sitting at desks, which can lead to health problems such as obesity and poor posture. 
Outdoor learning, on the other hand, encourages children to move around and engage in physical activities such as running, jumping, 
and climbing. This can help to improve their overall physical health and reduce the risk of health problems associated with a 
sedentary lifestyle.

In addition to the physical benefits, outdoor learning has also been shown to improve academic performance. 
Studies have found that children who participate in outdoor learning activities perform better in subjects such as science, math, 
and social studies. This may be due to the fact that outdoor environments provide a more hands-on and interactive learning experience, 
which can help children better understand and retain the material.

Finally, outdoor learning can also help to increase children's engagement in learning. Traditional classroom settings can sometimes 
be boring and uninspiring, which can lead to disinterest in learning. Outdoor learning, on the other hand, can provide a more stimulating 
and engaging learning experience, as children are able to explore and interact with the natural world around them.

Overall, outdoor learning has numerous benefits for children, including improved physical health, better academic performance, 
and increased engagement in learning. As such, it is an educational approach that should be embraced by educators and parents alike.



(choose ONE WORD ONLY from the passage)\\


Outdoor learning is an educational approach that focuses on teaching children in ANSWER environments.\\
Children who participate in outdoor learning activities perform better in subjects such as science, math, and ANSWER studies.\\
Traditional classroom settings can sometimes be boring and uninspiring, which can lead to disinterest in ANSWER.\\
Outdoor learning can provide a more stimulating and engaging learning experience, as children are able to explore 
and interact with the ANSWER world around them.\\

Do the following statements agree with the information given in Reading Passage 2?
write\\
YES if the statement agrees with the information\\
NO if the statement contradicts the information\\
NOT GIVEN if there is no information on this\\


Outdoor learning can help to reduce the risk of health problems associated with a sedentary lifestyle. (YES/NO/NOT GIVEN)\\
Outdoor learning has no effect on academic performance. (YES/NO/NOT GIVEN)\\
Outdoor learning is an educational approach that should be embraced by educators and parents alike. (YES/NO/NOT GIVEN)\\
Choose the correct letter, A, B, C or D\\

What is outdoor learning?\\
A. An educational approach that focuses on teaching children in outdoor environments.\\
B. An educational approach that focuses on teaching children in indoor environments.\\
C. An educational approach that focuses on teaching children through online courses.\\
D. An educational approach that focuses on teaching children through textbooks.\\

Why does outdoor learning promote physical activity?\\
A. Because children are encouraged to sit at desks for long periods of time.\\
B. Because outdoor environments provide a more hands-on and interactive learning experience.\\
C. Because outdoor learning encourages children to move around and engage in physical activities.\\
D. Because outdoor learning is a sedentary approach.\\

Which subjects can outdoor learning improve academic performance in?\\
A. History and geography\\
B. Science, math, and social studies\\
C. Literature and language\\
D. Art and music\\

Match each paragraph with its corresponding benefit of outdoor learning(2-4 paragraph).\\
A. Improved physical health\\
B. Better academic performance\\
C. Increased engagement in learning\\

\section{READING PASSAGE 2}
We are all in the same building, only the fire hasn't reached us yet.” So said a protester in Hunan province on November 27th. 
He was referring to a blaze three days earlier that killed ten people in an apartment block in Urumqi, the capital of Xinjiang. 
Many Chinese believe the country's covid restrictions contributed to the tragedy. So, in a rare moment of regional solidarity, 
hundreds of them took to the streets in cities across China, voicing their displeasure with the “zero-covid” policy.

The solidarity only goes so far, though. Neither Chinese officials nor the protesters mentioned the ethnicity of the fire victims. 
Most, if not all, are Uyghurs, members of a mostly Muslim minority who have suffered under a years long campaign of mass detention, 
forced labour and cultural erasure in the name of counter-terrorism. It is too dangerous for Uyghurs to be involved in the protests. 
But members of China's dominant ethnic group, the Han Chinese, show little sympathy for their broader plight, either 
because the Han are not aware of it, don't believe it is happening or think that addressing it would cross a red line set by the government.

This has led to awkward moments. At a vigil for the fire victims in Beijing on November 27th, one demonstrator said, 
“I am from Xinjiang, thank you all for coming!” Another replied pointedly: “We are all Chinese!” Abroad, where Uyghurs can protest without 
fear of arrest, things have got uncomfortable at times. There were two vigils in Amsterdam on November 28th, one led by Uyghurs, 
the other by Han Chinese. The two groups started arguing when the Uyghurs raised an East Turkestan flag, a symbol of independence. 

The Han protesters did not want to be associated with separatism; some criticised the Uyghurs for asserting that they were not Chinese. 
It was “frustrating” and “ridiculous”, says a Han participant who had hoped the groups could stand together. Han protesters are missing 
the bigger picture, says Abduweli Ayup, a Uyghur activist in Norway. He notes that the apartment block that caught fire was located in a 
Uyghur district of the city, close to where riots broke out in 2009. So the building was not only under a covid lockdown, it was also 
hemmed in by roadblocks put up in the name of security (a common feature in Uyghur areas). Videos showed water from fire trucks falling 
short of the blaze, apparently because they could not get close enough. A callous official said the victims had been “too weak” to save 
themselves. At least one survivor of the fire has since been detained, according to Mr Ayup

Some of China's longest and harshest lockdowns have been in border regions such as Xinjiang and Tibet, both of which have been 
largely sealed off for months. Residents have struggled with shortages of food and medicine—and with a lack of public attention. 
In October hundreds of anti-lockdown protesters clashed with police in Lhasa, the capital of Tibet, a situation ignored by Chinese media. 
Three Xinjiang residents were investigated in November after flooding a livestream of State Council proceedings with comments of 
“Urumqi Urumqi Urumqi”. The government has worked hard to keep these remote areas out of the public consciousness through censorship 
and policing. The fire in Urumqi changed that.

(choose ONE WORD ONLY from the passage)\\
a. On November 27th, hundreds of protesters took to the streets in cities across China, voicing their displeasure with 
the “zero-covid” policy, after a blaze three days earlier that killed ten people in an apartment block in ANSWER.\\
b. The majority of the fire victims in Urumqi are ANSWER.\\
c. Members of China's dominant ethnic group, the Han Chinese, show little sympathy for the Uyghurs' broader plight, 
either because the Han are not aware of it, don't believe it is happening or think that addressing 
it would cross a ANSWER set by the government.\\

Do the following statements agree with the information given in Reading Passage 2?
write\\
YES if the statement agrees with the information\\
NO if the statement contradicts the information\\
NOT GIVEN if there is no information on this\\

a. The fire in Urumqi was not related to the country's covid restrictions.\\
b. The Uyghurs are afraid to participate in the protests.\\
c. The Han Chinese sympathize with the Uyghurs' plight.\\
d. The two groups in Amsterdam, the Uyghurs and the Han Chinese, stood together without any disagreement. \\

Choose the correct letter\\
a. What did the protester in Hunan province mean when he said, “We are all in the same building, only the fire hasn't reached us yet”?\\
i. He was referring to a blaze that killed ten people in an apartment block in Urumqi, the capital of Xinjiang.\\
ii. He believed that the government's "zero-covid" policy contributed to the tragedy.\\
iii. He was showing solidarity with the Uyghurs, who have suffered under a yearslong campaign of mass detention, 
forced labour and cultural erasure.\\
iv. He was criticizing the Han Chinese for not being aware of or sympathizing with the Uyghurs' broader plight.\\

b. Why did some Han Chinese protesters criticize the Uyghurs at a vigil for the fire victims in Beijing?\\
i. The Uyghurs were raising an East Turkestan flag, a symbol of independence, which the Han protesters did not want to be associated with.\\
ii. The Uyghurs were protesting against the "zero-covid" policy.\\
iii. The Uyghurs were asserting that they were not Chinese, which offended the Han protesters.\\
iv. The Han protesters believed that the Uyghurs were responsible for the fire in Urumqi.\\

c. What was Abduweli Ayup's response to the Han Chinese protesters' criticism of the Uyghurs in Amsterdam?\\
i. He agreed with the Han protesters that the Uyghurs should not raise an East Turkestan flag.\\
ii. He criticized the Han protesters for not being aware of or sympathizing with the Uyghurs' broader plight.\\
iii. He thought that the Han protesters were right to distance themselves from the Uyghurs.\\
iv. He believed that the Han protesters were missing the bigger picture.\\

Match the paragraphs with their corresponding summaries:\\

A. The tragic fire in Urumqi and the protests it sparked\\
B. The lack of sympathy for Uyghurs among Han Chinese\\
C. The difficulties faced by Uyghurs in protesting abroad\\
D. The harsh lockdowns and lack of public attention in Xinjiang and Tibet\\

1.The tragic fire in Urumqi leads to protests across China, but the protesters and officials do not mention the ethnicity of the victims.\\
2.Members of China's dominant ethnic group, the Han Chinese, show little sympathy for the plight of Uyghurs.\\
3.Uyghurs face difficulties in protesting abroad, and Han Chinese protesters in Amsterdam disagree with their assertion of non-Chinese identity.\\
4.Residents of border regions such as Xinjiang and Tibet have struggled with harsh lockdowns and lack of public attention, but incidents such as the Urumqi fire bring attention to these areas.\\


\section{READING PASSAGE 3}
The world's largest genetic database, maintained by the Chinese company BGI, is at the center of a growing controversy. 
In recent years, BGI has amassed a collection of over 40 petabytes of genomic data, making it the largest of its kind in the world. 
However, concerns have been raised about how BGI is using this data, and who has access to it. 
Critics say that BGI has been using the data to further the Chinese government's interests, and that the company's 
practices are putting the privacy and security of individuals at risk.

The data in question comes from a variety of sources, including medical institutions, research centers, 
and direct-to-consumer genetic testing companies. BGI has been collecting this data through a variety of partnerships 
and collaborations. However, there are concerns that BGI has not been transparent about how it is using this data, 
or who has access to it. Some experts have raised concerns that the company may be using the data to develop bioweapons 
or other military applications.

BGI has denied these allegations, stating that the data is used only for scientific research and medical purposes, 
and that it is subject to strict privacy and security protocols. However, these assurances have not satisfied many critics, 
who point to China's history of using science and technology for military purposes.

The controversy over BGI's genetic database has led to calls for greater regulation and oversight of the use of genomic data. 
Some experts have called for a global framework to govern the collection, storage, and use of genetic data, to ensure that it is 
used ethically and in the public interest.

Others argue that the solution is not to restrict access to genetic data, but rather to ensure that it is used in a way that 
benefits society as a whole. This could involve greater transparency about how data is collected and used, and more involvement 
from the public in the decision-making process around genomic research.

Despite these concerns, BGI continues to expand its genetic database, and is now working on a project to sequence the genomes of every 
known species on Earth. The company has stated that this project will be used for conservation and biodiversity research, but critics 
are skeptical, given the company's past track record.

As the debate over BGI's genetic database continues, it is clear that the use of genomic data is a complex and contentious issue. 
While there are certainly risks associated with the collection and use of this data, there are also enormous potential benefits. 
It is up to regulators, researchers, and the public to work together to ensure that genetic data is used in a way that maximizes 
its benefits while minimizing its risks.\\

(choose ONE WORD ONLY from the passage)\\

BGI has amassed a collection of over 40 ANSWER of genomic data, making it the largest of its kind in the world.\\
Critics say that BGI has been using the data to further the Chinese government's interests, and that the company's practices are putting the privacy and security of individuals at ANSWER.\\
Some experts have raised concerns that the company may be using the data to develop ANSWER or other military applications.\\
The controversy over BGI's genetic database has led to calls for greater regulation and oversight of the use of ANSWER data.\\
Do the following statements agree with the information given in Reading Passage 2?
write\\
YES if the statement agrees with the information\\
NO if the statement contradicts the information\\
NOT GIVEN if there is no information on this\\

1.BGI's genetic database is currently the largest of its kind in the world. (YES/NO/NOT GIVEN)\\
2.BGI has been transparent about how it is using the genomic data it has collected. (YES/NO/NOT GIVEN)\\
3.The controversy over BGI's genetic database has not led to calls for greater regulation and oversight of the use of genomic data. (YES/NO/NOT GIVEN)\\
4.Critics are confident that BGI's project to sequence the genomes of every known species on Earth will be used for conservation and biodiversity research. (YES/NO/NOT GIVEN)\\

Choose the correct letter\\

What is the main concern about BGI's genetic database?\\
A. It is too expensive to access for most researchers.\\
B. It is being used for military purposes by the Chinese government.\\
C. It is not large enough to be useful for most research projects.\\
D. It is not subject to strict privacy and security protocols.\\

What is the controversy surrounding BGI's genetic database?\\
A. The data is being used to develop bioweapons.\\
B. The company is not transparent about how it is using the data.\\
C. The data is too difficult to access for most researchers.\\
D. The data is being used to develop new treatments for genetic diseases.\\

What is the solution to the controversy over the use of genomic data, according to some experts?\\
A. Restrict access to genetic data.\\
B. Ensure that genetic data is used in a way that benefits society as a whole.\\
C. Use genetic data to develop new bioweapons.\\
D. Allow private companies to collect and sell genetic data for profit.\\

According to the passage, what is BGI's project to sequence the genomes of every known species on Earth intended for?\\
A. Military applications\\
B. Profit-making\\
C. Conservation and biodiversity research\\
D. Medical research\\


What is the main concern regarding the collection and use of genomic data, according to the passage?\\
A. There are no benefits associated with genomic data collection and use.\\
B. Genetic data is too expensive to collect and use.\\
C. There are potential risks associated with the collection and use of genetic data.\\
D. Genomic data collection and use is subject to too much regulation and oversight.\\

\section{Answer}
\subsection{READING PASSAGE 1}

sedentary;
science, math, social studies;
stimulating;
YES;
YES;
YES;
NOT GIVEN;
YES;
NO;
A
B
C
D
paragraph 1: A;
paragraph 2: C;
paragraph 3: B
\subsection{READING PASSAGE 2}
Urumqi;
Uyghurs;
Line;
NOT GIVEN;
YES;
NO;
NO;
iii;
i;
iv;
A-1;
B-2;
C-3;
D-4
\section{READING PASSAGE 3}
petabytes;
risk;
bioweapons;
genomic;
Earth;
YES;
NOT GIVEN;
NO;
NOT GIVEN;
BABCC

\end{spacing}{}

\end{document}